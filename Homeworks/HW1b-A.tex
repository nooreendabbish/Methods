% Created 2015-01-29 Thu 22:29
\documentclass[11pt]{article}
\usepackage[utf8]{inputenc}
\usepackage[T1]{fontenc}
\usepackage{fixltx2e}
\usepackage{graphicx}
\usepackage{longtable}
\usepackage{float}
\usepackage{wrapfig}
\usepackage{soul}
\usepackage{textcomp}
\usepackage{marvosym}
\usepackage{wasysym}
\usepackage{latexsym}
\usepackage{amssymb}
\usepackage{hyperref}
\tolerance=1000
\usepackage{methodshw}
\usepackage{booktabs}
\providecommand{\alert}[1]{\textbf{#1}}

\title{8004 Homework 1}
\author{Nooreen Dabbish}
\date{\today}
\hypersetup{
  pdfkeywords={},
  pdfsubject={},
  pdfcreator={Emacs Org-mode version 7.9.3f}}

\begin{document}

\maketitle



\section{Write out the following models of elementary/intermediate statistical analysis in the matrix form}
\label{sec-1}

\[ \mathbf{Y} = \mathbf{X\beta} +\mathbf{\epsilon} \]
\subsection{A one-variable quadratic polynomial regression model}
\label{sec-1-1}

\[
y_i = \alpha_0 + \alpha_1x_i + \alpha_2 x_i^2 + \epsilon_i \textrm{
for }   (i = 1,2,\ldots,5).
\]




\[
\mathbf{y} = 
\begin{pmatrix}{}
  y_1 \\ 
  y_2 \\ 
  y_3 \\ 
  y_4 \\ 
  y_5 \\ 
  \end{pmatrix},\,
\mathbf{X} = 
\begin{pmatrix}{}
  1 & x_1 & x^2_1 \\ 
  1 & x_2 & x^2_2 \\ 
  1 & x_3 & x^2_3 \\ 
  1 & x_4 & x^2_4 \\ 
  1 & x_5 & x^2_5 \\ 
  \end{pmatrix},\,
\mathbf{\beta} = 
\begin{pmatrix}{} 
  \alpha_0 \\ 
  \alpha_1 \\ 
  \alpha_2 \\
  \end{pmatrix}, 
\mathbf{\epsilon} =
\begin{pmatrix}{}
  \epsilon_1 \\ 
  \epsilon_2 \\ 
  \epsilon_3 \\ 
  \epsilon_4 \\ 
  \epsilon_5 \\ 
  \end{pmatrix}
\]

\textbf{y} = \textbf{X$\beta$} + \textbf{$\epsilon$}  in this model is therefore:

\[
\begin{pmatrix} y_1 \\ y_2 \\ y_3 \\ y_4\\ y_5\end{pmatrix}
 = \begin{pmatrix} 
                            1 & x_1 & x_1^2 \\ 
                            1 & x_2 & x_2^2 \\ 
                            1 & x_3 & x_3^2 \\ 
                            1 & x_4 & x_4^2 \\ 
                            1 & x_5 & x_5^2 \end{pmatrix} \begin{pmatrix} \alpha_0 \\ \alpha_1 \\ \alpha_2
\end{pmatrix} + \begin{pmatrix} \epsilon_1 \\ \epsilon_2 \\
\epsilon_3 \\ \epsilon_4 \\ \epsilon_5 \end{pmatrix}
\]
\subsection{A two-factor ANCOVA model without interactions}
\label{sec-1-2}

\[
y_{ijk} = \mu + \alpha_i + \beta_j + \gamma(x_{ijk} - \bar{x}) +
\epsilon_{ijk}\,\, \textrm{ for } i= 1, 2,\,j=1,2,\,\textrm{ and }k
=1,2.
\]

This model describes an 8-dimensional data space, where the column of
centered x values may be calculated as follows:

\[ 
\mathbf{y} = \begin{pmatrix} y_{111} \\ y_{112} \\  y_{121}
\\ y_{122} \\ y_{211} \\ y_{212} \\ y_{221} \\ y_{222}
\end{pmatrix},\, 
\mathbf{\epsilon} = \begin{pmatrix} \epsilon_{111} \\ \epsilon_{112} \\  \epsilon_{121}
\\ \epsilon_{122} \\ \epsilon_{211} \\ \epsilon_{212}
\\ \epsilon_{221} \\ \epsilon_{222} \end{pmatrix},\, 
\mathbf{x}_c = (\mathbf{I}-
\frac{1}{n} \mathbf{J}) \begin{pmatrix} x_{111} \\ x_{112} \\  x_{121}
\\ x_{122} \\ x_{211} \\ x_{212} \\ x_{221} \\ x_{222} \end{pmatrix}
= \begin{pmatrix} x_{111} -\bar{x} \\ x_{112} - \bar{x} \\  x_{121} - \bar{x}
\\ x_{122} -\bar{x} \\ x_{211} -\bar{x} \\ x_{212} -\bar{x}
\\ x_{221} - \bar{x} \\ x_{222} -\bar{x} \end{pmatrix}\]


The design matrix \textbf{X}  and regression coefficient vector \textbf{$\beta$} are given by: 

\[ \mathbf{X} =
\begin{pmatrix}{}
  1  & 1  & 0  & 1  & 0  & x_{111} -\bar{x}\\ 
  1  & 1  & 0  & 1  & 0  & x_{112} -\bar{x}\\ 
  1  & 1  & 0  & 0  & 1  & x_{121} -\bar{x}\\ 
  1  & 1  & 0  & 0  & 1  & x_{122} -\bar{x}\\ 
  1  & 0  & 1  & 1  & 0  & x_{211} -\bar{x}\\ 
  1  & 0  & 1  & 1  & 0  & x_{212} -\bar{x}\\ 
  1  & 0  & 1  & 0  & 1  & x_{221} -\bar{x}\\ 
  1  & 0  & 1  & 0  & 1  & x_{222} -\bar{x}\\ 
  \end{pmatrix},\,
\mathbf{\beta} = \begin{pmatrix} \mu \\ \alpha_1 \\ \alpha_2
\\ \beta_1 \\ \beta_2 \\ \gamma \end{pmatrix}
\]

Putting these together gives the model:
\[
\begin{pmatrix} y_{111} \\ y_{112} \\  y_{121}
\\ y_{122} \\ y_{211} \\ y_{212} \\ y_{221} \\ y_{222}
\end{pmatrix} =  
\begin{pmatrix}{}
  1  & 1  & 0  & 1  & 0  & x_{111} -\bar{x}\\ 
  1  & 1  & 0  & 1  & 0  & x_{112} -\bar{x}\\ 
  1  & 1  & 0  & 0  & 1  & x_{121} -\bar{x}\\ 
  1  & 1  & 0  & 0  & 1  & x_{122} -\bar{x}\\ 
  1  & 0  & 1  & 1  & 0  & x_{211} -\bar{x}\\ 
  1  & 0  & 1  & 1  & 0  & x_{212} -\bar{x}\\ 
  1  & 0  & 1  & 0  & 1  & x_{221} -\bar{x}\\ 
  1  & 0  & 1  & 0  & 1  & x_{222} -\bar{x}\\ 
  \end{pmatrix}
\begin{pmatrix} \mu \\ \alpha_1 \\ \alpha_2
\\ \beta_1 \\ \beta_2 \\ \gamma \end{pmatrix} + 
\begin{pmatrix} \epsilon_{111} \\ \epsilon_{112} \\  \epsilon_{121}
\\ \epsilon_{122} \\ \epsilon_{211} \\ \epsilon_{212}
\\ \epsilon_{221} \\ \epsilon_{222} \end{pmatrix}
\]

*  \section{Use eigen() function in R to compute the eigenvalues and eigenvectors of} 

\[
\mathbf{V} =
% latex table generated in R 3.0.2 by xtable 1.7-4 package
% Thu Jan 29 22:29:43 2015
\begin{pmatrix}{}
  3 & -1 & 1 \\ 
  -1 & 5 & -1 \\ 
  1 & -1 & 3 \\ 
  \end{pmatrix} 
\]

Then use R to find and ``inverse square root'' of this matrix.
That is, find a symmetric matrix $\mathbf{W}$ such that
$\mathbf{WW}=\mathbf{V}^{-1}$.
\subsection{Eigenvalues and Eigenvectors}
\label{sec-2-1}


I ran the following R-code using \verb~eigen()~ eigen to calculate eigenvalues
and eigenvectors. A small function \verb~lvector()~ calls \verb~xtable()~ and \verb~print()~
in order to generate latex output below (see appendix). 

\begin{verbatim}
V <- matrix(c(3, -1, 1, -1, 5, -1, 1, -1, 3), 3,3, byrow=TRUE)
lvector(as.matrix(eigen(V)$values), dig=0)
for (i in 1:3){
lvector(as.matrix(eigen(V)$vectors[,i]))
}
\end{verbatim}

The eigenvalues are 
\[
\mathbf{\lambda} = \begin{pmatrix} \lambda_1
\\ \lambda_2 \\ \lambda_3 \end{pmatrix} =
\begin{pmatrix}{}
    6 \\ 
    3 \\ 
    2 \\ 
  \end{pmatrix}
\]

With corresponding eigenvectors:
\[
\mathbf{e_1} = 
 % latex table generated in R 3.0.2 by xtable 1.7-4 package
% Mon Jan 26 13:52:05 2015
\begin{pmatrix}{}
  0.41 \\ 
  -0.82 \\ 
  0.41 \\ 
  \end{pmatrix},\,
\mathbf{e_2} =
% latex table generated in R 3.0.2 by xtable 1.7-4 package
% Mon Jan 26 13:52:05 2015
\begin{pmatrix}{}
  0.58 \\ 
  0.58 \\ 
  0.58 \\ 
  \end{pmatrix},\,
\mathbf{e_3} =
% latex table generated in R 3.0.2 by xtable 1.7-4 package
% Mon Jan 26 13:52:05 2015
\begin{pmatrix}{}
  0.71 \\ 
  0.00 \\ 
  -0.71 \\ 
  \end{pmatrix}.
\]
\subsection{Inverse Square Root}
\label{sec-2-2}


In order to find the ``inverse square root'', spectral
decomposition was performed, as outlined below. \textbf{U} is a matrix made up
of the eigenvectors of \textbf{V} in columns, and \textbf{D} is a matrix with \textbf{V}'s
eigenvalues along the diagonal. \textbf{V} and \textbf{V$^{\mathrm{-1}}$} have the same
eigenvectors, denoted u$_i$ below, and a square non-singular matrix with eigenvalues $\lambda$$_i$'s
will have an inverse with eigenvalues given by $\lambda$$_i$$^{\mathrm{-1}}$.

\begin{align*}
V &= U D U^T = \sum_{i=1}^{n} \lambda_i u_i u_i^T\\
V^{1/2} &= U D^{1/2} U^T = \sum_{i=1}^{n} \lambda_i^{1/2} u_i u_i^T\\
V^{-1} &= U D^{-1} U^T = \sum_{i=1}^{n} \lambda_i^{-1} u_i u_i^T\\
V^{-1/2} &= U D^{-1/2} U^T = \sum_{i=1}^{n} \lambda_i^{-1/2} u_i u_i^T
\end{align*}


\begin{verbatim}
V_inv = solve(V)
C = as.matrix(eigen(V_inv)$vectors)
D_sqrt = diag(lapply(eigen(V_inv)$values, sqrt))
W = C%*%D_sqrt%*%t(C)
lvector(W, dig=4)
\end{verbatim}

\[
\mathbf{W} =
\begin{pmatrix}{}
  0.6140 & 0.0564 & -0.0931 \\ 
  0.0564 & 0.4646 & 0.0564 \\ 
  -0.0931 & 0.0564 & 0.6140 \\ 
  \end{pmatrix}
\]

And a comparison between \textbf{WW} and \textbf{V$^{\mathrm{-1}}$}:


\begin{verbatim}
Prod = W%*%W
lvector(Prod)
lvector(solve(V))
\end{verbatim}

\[
\mathbf{WW} = 
\begin{pmatrix}{}
  0.39 & 0.06 & -0.11 \\ 
  0.06 & 0.22 & 0.06 \\ 
  -0.11 & 0.06 & 0.39 \\ 
  \end{pmatrix},\,
\mathbf{V^{-1}} =
\begin{pmatrix}{}
  0.39 & 0.06 & -0.11 \\ 
  0.06 & 0.22 & 0.06 \\ 
  -0.11 & 0.06 & 0.39 \\ 
  \end{pmatrix}
\]
\section{Consider the matrices}
\label{sec-3}





\[ \mathbf{A} = \begin{pmatrix}{} 4 & 4.001 \\ 4.001 & 4.002
\\ \end{pmatrix}\,\textrm{ and }\mathbf{B} = \begin{pmatrix}{} 4 &
4.001 \\ 4.001 & 4.002001 \\ \end{pmatrix}. \]

Obviously, these matrices are nearly identical. Use R and compute the
determinants and inverses of these matrices. (Even though the original
two matrices are nearly the same, $\mathbf{A}^{-1} \approx
-3\mathbf{B}^{-1}$. This shows that small changes in the elements of
nearly singular matrices can have big effects on some matrix
operations.)
\subsection{Determinants and Inverses.}
\label{sec-3-1}


Both determinants were determined using \verb~det()~ in R and are nearly zero. The determinant of A is \texttt{-1.00000000513756e-06} 
and the determinant of B is \texttt{2.99999999764467e-06}.

\[
\mathbf{A}^{-1} = 
\begin{pmatrix}{}
-4001999.98 & 4000999.98 \\ 
4000999.98 & -3999999.98  \\ 
\end{pmatrix}\, \textrm{ and }\, 
\mathbf{B}^{-1} = 
\begin{pmatrix}{} 
1334000.33 & -1333666.67 \\ 
-1333666.67 &  1333333.33 \\ 
\end{pmatrix}
\]

\#\$\$ 

\[
-3\mathbf{B}^{-1} = \begin{pmatrix}{}
-4002001.00 & 4001000.00 \\ 4001000.00 & -4000000.00
\\ \end{pmatrix}
\]
\section{Write an R function to conduct projection, e.g. with the name project()}
\label{sec-4}

The input is the given design matrix $\mathbf{X}$, and the output is
the projection matrix $\mathbf{P_X}$ for projecting a vector onto the
column space of $\mathbf{X}$.

In the following code, I define a function \verb~project()~ which accepts a
matrix as an input and uses the \verb~t()~ and \verb~ginv()~ functions to
calculate the transpose and inverse. $\mathbf{P_X} = \mathbf{X(X^T
X)^{-}X^T}$ is returned. I also ran a few tests, first using the
matrix $\mathbf{V}$ defined in problem 2 to make sure the projection
is symmetric and idempotent. I also tested the results of scalar and
$\mathbf{0}$ (matrix of all zero) inputs.
 

\begin{verbatim}
library(MASS) 
project <- function (X) {X%*%(ginv(t(X)%*%X))%*%t(X)}
\end{verbatim}



\begin{verbatim}
lvector(project(V))
\end{verbatim}

\[ \begin{pmatrix}{} 1.00 & 0.00 & 0.00 \\ -0.00 & 1.00 & -0.00
\\ 0.00 & 0.00 & 1.00 \\ \end{pmatrix} \]


\begin{verbatim}
lvector(project(V)%*%project(V))
\end{verbatim}

\[ \begin{pmatrix}{} 1.00 & 0.00 & 0.00 \\ -0.00 & 1.00 & -0.00
\\ 0.00 & 0.00 & 1.00 \\ \end{pmatrix} \]


\begin{verbatim}
lvector(t(project(V)))
\end{verbatim}

\[ \begin{pmatrix}{} 1.00 & -0.00 & 0.00 \\ 0.00 & 1.00 & 0.00 \\ 0.00
& -0.00 & 1.00 \\ \end{pmatrix} \]


\begin{verbatim}
lvector(project(3))
\end{verbatim}

\[ \begin{pmatrix}{} 1.00
\\ \end{pmatrix} \]



\begin{verbatim}
lvector(project(matrix(c(0,0,0,0),2,2)))
\end{verbatim}

\[ \begin{pmatrix}{} 0.00 & 0.00 \\ 0.00 & 0.00 \\ 
\end{pmatrix} \]
\section{Consider the (non-full-rank) two-way ``effect model'' with interactions in the Example (d) in lecture.}
\label{sec-5}
\subsection{Determine which of the parametric functions below are estimable:}
\label{sec-5-1}

\[
\alpha_1,\alpha_2 - \alpha_a, \mu + \alpha_1+ \beta_1+\delta_{11},
\delta_{12}, \delta_{12} - \delta_{11} - (\delta_{22} - \delta_{21})
\]
For those that are estimable, find $\mathbf{c}^T
(\mathbf{X}^{T}\mathbf{X})^{-}\mathbf{X}^{T}$, such that  $\mathbf{c}^T
(\mathbf{X}^{T}\mathbf{X})^{-}\mathbf{X}^{T}\mathbf{Y}$ produces the
estimate of $\mathbf{C}^{T}\mathbf{\beta}$.

\rule{0.8\textwidth}{.4pt}

Example (d) described a two-way model to study trees of types A and B
treated with either old or new fungicide. Each response variable,
$y_{ijk}$ represents the response of variety i to fungicide j of
the tree with index k, where $i=1,2$, $j=1,2$, and $k=1,2$. 

The model, $\mathbf{Y}=\mathbf{X\beta}+ \mathbf{\epsilon}$ in example (d) is:



\[
\begin{pmatrix} y_{111} \\ y_{112} \\  y_{121}
\\ y_{122} \\ y_{211} \\ y_{212} \\ y_{221} \\ y_{222}
\end{pmatrix} =  
\begin{pmatrix}{}
  1 & 1 & 0 & 1 & 0 & 1 & 0 & 0 & 0 \\ 
  1 & 1 & 0 & 1 & 0 & 1 & 0 & 0 & 0 \\ 
  1 & 1 & 0 & 0 & 1 & 0 & 1 & 0 & 0 \\ 
  1 & 1 & 0 & 0 & 1 & 0 & 1 & 0 & 0 \\ 
  1 & 0 & 1 & 1 & 0 & 0 & 0 & 1 & 0 \\ 
  1 & 0 & 1 & 1 & 0 & 0 & 0 & 1 & 0 \\ 
  1 & 0 & 1 & 0 & 1 & 0 & 0 & 0 & 1 \\ 
  1 & 0 & 1 & 0 & 1 & 0 & 0 & 0 & 1 \\ 
  \end{pmatrix}
\begin{pmatrix}
\mu \\ \alpha_1 \\ \alpha_2 \\ 
\beta_1 \\ \beta_2 \\ 
\delta_11 \\ \delta_12 \\ \delta_21 \\ \delta_22
\end{pmatrix} +
\begin{pmatrix} \epsilon_{111} \\ \epsilon_{112} \\  \epsilon_{121}
\\ \epsilon_{122} \\ \epsilon_{211} \\ \epsilon_{212}
\\ \epsilon_{221} \\ \epsilon_{222} \end{pmatrix}
\]

The only regression coefficients or combinations that can be estimated 
can be written in the form $\mathbf{c^T\beta}$, where the vector 
$\mathbf{c}$ is in the row space of the design matrix $\mathbf{X}$ 
(equivalent to the column space of $\mathbf{X^T}$).

There are 5 parametric functions in this question:

\begin{enumerate}
\item $\alpha_1,\, c^T=(0,1,0,0,0,0,0,0,0)$. This is not in the column
    space of $\mathbf{X^T}$, so it is not estimable.
\item $\alpha_2 - \alpha_1,\,c^T=(0,-1,1,0,0,0,0,0,0)$ This is not in the column
    space of $\mathbf{X^T}$, so it is not estimable.
\item $\mu + \alpha_1 + \beta_1 +\delta_{11}, c^T=(1,1,0,1,0,1,0,0,0)$
    This is in the column space of $\mathbf{X^T}$, so it is estimable.
\item $\delta_{12},\,c^T=(0,0,0,0,0,0,1,0,0)$.  Not in the column space
    of $\mathbf{X^T}$ and therefore not estimable.
\item $\delta_{12} - \delta_{11} - (\delta_{22} -
    \delta_{21}),\,c^T=(0,0,0,0,0,-1,1,1,-1)$. This is in the column
    space of $\mathbf{X^T}$, so it is estimable.
\end{enumerate}

We can use R to find $\mathbf{c^T}(X^T X)^{-}X^T$ for $\mu +
\alpha_1 + \beta_1 +\delta_11$
 and $\delta_12 - \delta_11 - (\delta_22 -
    \delta_21)$.


\begin{verbatim}
X <- matrix(c(rep(1,8), rep(1,4), rep(0,8), rep(1,4), 
              rep(c(1,1,0,0),2), rep(c(0,0,1,1),2),
              rep(c(1,1,rep(0,8)),3), 1,1),nrow=8,ncol=9, byrow=FALSE)
ct3 <- matrix(c(1,1,0,1,0,1,0,0,0),nrow=1)
ct5 <- matrix(c(0,0,0,0,0,-1,1,1,-1), nrow=1)
lvector(ct3%*%ginv(t(X)%*%X)%*%t(X))
lvector(ct5%*%ginv(t(X)%*%X)%*%t(X))
\end{verbatim}

\[
\mathbf{c_{\mu +\alpha_1 + \beta_1 +\delta_11}^T (X^TX)^{-}X^T} =
\begin{pmatrix}{}
  0.50 & 0.50 & -0.00 & -0.00 & -0.00 & -0.00 & 0.00 & 0.00 \\ 
  \end{pmatrix}
\]

and
\[
\mathbf{c_{\delta_{12} - \delta_{11} - (\delta_{22} -
    \delta_{21})}^T  (X^TX)^{-}X^T} =
\begin{pmatrix}{}
  -0.50 & -0.50 & 0.50 & 0.50 & 0.50 & 0.50 & -0.50 & -0.50 \\ 
  \end{pmatrix}
\]
\subsection{For the parameter vector $\beta$ written in the order used in class, consider the hypothesis $H_0$ : $\mathbf{C\beta} = \mathbf{0}$ for}
\label{sec-5-2}



\[
\mathbf{C} =
\begin{pmatrix}{}
    0 &   1 &  -1 &   0 &   0 &   0 &   0 &   0 &   0 \\ 
    0 &   0 &   0 &   0 &   0 &   1 &  -1 &  -1 &   1 \\ 
  \end{pmatrix}
\]

Is this hypothesis testable? Explain.


No, this hypothesis is not fully testable. This hypothesis asks whether the
parametric function $\alpha_1 - \alpha_2$ (represented by the first 
row of \textbf{C}) is zero and the parametric function 
$\delta_11 - \delta_12 - \delta_21 + \delta_22$ is zero (second row). 
The second row of $\mathbf{C}$ is in the
column space of $\mathbf{X^T}$ so it would be testable. However, the
first row is not, the function $\alpha_1 - \alpha_2$ is not estimable
and therefore not testable.
\section{Appendix: Tangled R-code}
\label{sec-6}

\lstinputlisting{HW1a.R}

\end{document}
