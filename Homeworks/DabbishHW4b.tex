% Created 2015-02-18 Wed 19:08
\documentclass[11pt]{article}
\usepackage[utf8]{inputenc}
\usepackage[T1]{fontenc}
\usepackage{fixltx2e}
\usepackage{graphicx}
\usepackage{longtable}
\usepackage{float}
\usepackage{wrapfig}
\usepackage{soul}
\usepackage{textcomp}
\usepackage{marvosym}
\usepackage{wasysym}
\usepackage{latexsym}
\usepackage{amssymb}
\usepackage{hyperref}
\tolerance=1000
\usepackage{methodshw}
\usepackage{booktabs}
\providecommand{\alert}[1]{\textbf{#1}}

\title{8004 Homework 4}
\author{Nooreen Dabbish}
\date{\today}
\hypersetup{
  pdfkeywords={},
  pdfsubject={},
  pdfcreator={Emacs Org-mode version 7.9.3f}}

\begin{document}

\maketitle




\section{Problem 1 In the context of Problem 2 of Homework Assignment 3, use R matrix calculations to do the following in the (non-full-rank) Gauss-Markov normal linear model}
\label{sec-1}
\subsection{Find 90\% two-sided confidence limits for $\sigma$.}
\label{sec-1-1}


The model described in HW3, Problem 2 in 
$\mathbf{Y}=\mathbf{X\beta}+\epsilon$ matrix form is:

\[
\begin{pmatrix}
y_{11} \\ y_{12}\\ y_{21}\\ y_{31}\\ y_{41}\\ y_{42}
\end{pmatrix} = 
\begin{pmatrix} 
2\\ 1\\ 4\\ 6\\ 3\\ 5
\end{pmatrix} = 
\begin{pmatrix}
1 & 1 & 0 & 0 & 0 \\
1 & 1 & 0 & 0 & 0 \\
1 & 0 & 1 & 0 & 0 \\
1 & 0 & 0 & 1 & 0 \\
1 & 0 & 0 & 0 & 1 \\
1 & 0 & 0 & 0 & 1 \\
\end{pmatrix}  
\begin{pmatrix}
\mu \\ \tau_1 \\ \tau_2 \\ \tau_3 \\ \tau_4 
\end{pmatrix} + 
\begin{pmatrix}
\epsilon_{11} \\ \epsilon_{12}\\ \epsilon_{21}\\ \epsilon_{31}\\ \epsilon_{41}\\ \epsilon_{42}
\end{pmatrix}
\]

Because the problem statement says this is a Gauss-Markov normal
linear model, we know that \textbf{Y} $\sim$ N(\textbf{X$\beta$},$\sigma$$^2$ \textbf{I}).
\subsubsection{Interval for $\sigma$ using \textbf{I}}
\label{sec-1-1-1}

The Gauss-Markov normal linear model assumes that the var(\textbf{Y}) =
$\sigma$$^2$ \textbf{I}, and in this case we are able to solve for SSE directly
from $\mathbf{\hat{Y}}$ and \textbf{X}.


\begin{verbatim}
Yhat <- X %*% ginv(t(X) %*% X) %*% t(X) %*% Y

SSE1a3 <- t(Y-Yhat) %*% (Y-Yhat)

lowerchi <- qchisq(.05, df=(length(Y) - qr(X)$rank))
upperchi <- qchisq(.95, df=(length(Y) - qr(X)$rank))
\end{verbatim}

For the Gauss-Markov linear model of HW3 Problem 2, we found an SSE of \texttt{2.5} and two-sided 90\% confidence
limits for $\sigma$ of \texttt{0.5656} <
$\sigma$ < \texttt{2.6656}.
\subsection{Find 90\% two-sided confidence limits for $\mu$ + $\tau$$_2$.}
\label{sec-1-2}
\label{MuTau2}

The following provides 90\% confidence limits for $\mu$ + $\tau$$_2$ in the
Gauss-Markov model first, where \textbf{Y} $\sim$ N$_6$(\textbf{X$\beta$}, $\sigma$$^2$ \textbf{I}) and then in the GLS cases with var(\textbf{Y}) =
$\sigma$$^2$ \textbf{V$_1$} and var(\textbf{Y}) = $\sigma$$^2$ \textbf{V$_2$}.
\subsection{Find 90\% two-sided confidence limits for $\tau$$_1$ - $\tau$$_2$.}
\label{sec-1-3}
\label{Tau1Tau2}

Proceeding as in \hyperref[MuTau2]{part b}, here $\tau$$_1$ - $\tau$$_2$ = \textbf{a}'\textbf{$\beta$} = $(0, 1, -1,
0, 0) \begin{pmatrix} \mu \\ \tau_1 \\ \tau_2
\\ \tau_3 \\ \tau_4 \end{pmatrix}$. Note that the quantile for
\emph{t$_{\alpha/2}$} and value for \emph{s} are calculated above.


\begin{verbatim}

a_1c = matrix(c(0,1,-1,0,0))

quad_1c <- sqrt(t(a_1c) %*% ginv(t(W)%*%W) %*% a_1c)
upper1c <- t(a_1c) %*% Bhat_1b - t_1b * s_1b * quad_1c
lower1c <- t(a_1c) %*% Bhat_1b + t_1b * s_1b * quad_1c
\end{verbatim}

We find that the 90\% confidence limits for $\tau$$_1$ - $\tau$$_2$ are from \texttt{-12.8237} to \texttt{7.8237}. 
\subsection{Find a \emph{p}-value for testing the null hypothesis $H_0 : \tau_1 - \tau_2 = 0$ vs \emph{H$_a$} : not \emph{H$_0$}.}
\label{sec-1-4}
\label{Tau1Tau2H0}
\subsubsection{General Linear Hypothesis Test}
\label{sec-1-4-1}

\emph{The general linear hypothesis test} is the following F test for
\emph{H$_0$} : \textbf{C$\beta$} = \textbf{0} verus \emph{H$_1$} : \textbf{C$\beta$} $\neq$ \textbf{0}, given \textbf{y}
$\sim$ N$_n$(\textbf{X$\beta$},$\sigma$$^2$ \textbf{I}), \textbf{C} \emph{q} x (/k/+1), rank(\textbf{C}) = q, with SSH = the sum of squares due to
the hypothesis or due to \textbf{C$\beta$}. Note that 

\(\frac{SSH}{\sigma^2} = \frac{(\mathbf{C\hat{\beta}})'[\mathbf{C(X'X)^{-1}C}']^{-1}\mathbf{C\hat{\beta}}}{\sigma^2}
\sim
\chi^2(q,\frac{(\mathbf{C\beta})'[\mathbf{C(X'X)^{-1}C}']^{-1}\mathbf{C\beta}}{2\sigma^2})\)

\noindent and

\(\frac{SSE}{\sigma^2} = \frac{\mathbf{y}'[\mathbf{I} -
\mathbf{X(X'X)^{-1}X}']\mathbf{y}}{\sigma^2} \sim \chi^2(n-k-1).\)

Taking the ratio gives us our test statistic:
$$F = \frac{SSH/q}{SSE/(n-k-1)}$$

\begin{itemize}
\item If \emph{H$_0$} : \textbf{C$\beta$} = \textbf{0} is false, F $\sim$ F(q,n-k-1,$\lambda$), where
  $\lambda =
  \frac{(\mathbf{C\beta})'[\mathbf{C(X'X)^{-1}C}']^{-1}\mathbf{C\beta}}{2\sigma^2})$.
\item Notice that if \textbf{C$\beta$} = \textbf{0} is true, $\lambda$ defined above = 0, giving F
  $\sim$ F(q, n-k-1).
\end{itemize}
\subsubsection{\emph{p}-value from the F statistic}
\label{sec-1-4-2}


We need to find the F statistic described above. Here \textbf{C} is \textbf{a}' from \hyperref[Tau1Tau2]{above}, \textbf{a}'=(0,1,-1,0,0), and \textbf{C} is 1 x 5 of
rank 1, so q = 1. Note also that n=6, k=4, n-k-1=1.


\begin{verbatim}

SSH <- t(t(a_1c) %*% Bhat_1b) %*% ginv(t(a_1c)%*%ginv(t(W)%*%W)%*%a_1c)%*%t(a_1c)%*%Bhat_1b

p_1d <- pf(SSH/SSE, 1, 1, lower.tail=FALSE)
\end{verbatim}

The \emph{p}-value obtained was \texttt{0.7048}. This is the probability that the central F
distribution exceeds the observed F. This suggests that we should accept the null
hyposthesis.
\subsection{Find 90\% two-sided predition limits for the sample mean of /n/=10 future observations from the first set of conditions.}
\label{sec-1-5}
\subsubsection{A t statistic for prediction}
\label{sec-1-5-1}


Consider future observation y$_0$, y$_0$ = \textbf{x$_0$}' $\beta$ + $\epsilon$$_0$ with
$\hat{y}_0 = \mathbf{x_0'\hat{\beta}}$, where $\hat{y_0}$ is computed
from \emph{n} observations and y$_0$ is obtained independently. We find that
$E(y_0 - \hat{y_0}) = 0$ and 


$var(y_0 - \hat{y}_0) = var(\epsilon_0) +
var(\mathbf{x_0'\hat{\beta}}) = \sigma^2[1+
\mathbf{x_0'(X'X)^{-1}x_0}]$, where 
$\widehat{var(y - \hat{y}_0)} = s^22[1+ \mathbf{x_0'(X'X)^{-1}x_0}]$. Because of the independence of \emph{s$^2$}
and \emph{y$_0$} and $\hat{y}_0$, we have the following t statistic:

$$t = \frac{y_0 - \hat{y}_0 - 0}{s\sqrt{1+
\mathbf{x_0'(X'X)^{-1}x_0}} \sim t(n-k-1)}$$



Therefore,

$$P = \left[ -t_{\alpha/2,n-k-1} \leq \frac{y_0 - \hat{y}_0 - 0}{s\sqrt{1+
\mathbf{x_0'(X'X)^{-1}x_0}}} \leq t_{alpha/2,n-k-a}\right] = 1 -
\alpha$$

Re-arranging in terms of $\mathbf{x_0'\hat{\beta}} = \hat{y}_0$ gives:

$$\mathbf{x_0'\hat{\beta}} \pm t_{\alpha/2,n-k-1}s\sqrt{1+
\mathbf{x_0'(X'X)^{-1}x_0}}.$$
\subsection{Find 90\% two-sided prediction limits for the difference between a pair of future values, one from the first set of conditions (i.e. with mean $\mu$ + $\tau$$_1$) and one from the second set of conditions (i.e. with mean $\mu$ + $\tau$$_2$).}
\label{sec-1-6}
\subsection{Find a \emph{p}-value for testing the following: What is the practical interpretation of this test?}
\label{sec-1-7}

\( H_0 : \begin{pmatrix} 0 & 1 & -1 & 0 & 0 \\ 0 & 1 & 0 & -1 & 0 \\ 0 & 1 & 0 & 0 & -1 \end{pmatrix} \begin{pmatrix} \mu \\ \tau_1 \\ \tau_2 \\ \tau_3 \\ \tau_4 \end{pmatrix} = \begin{pmatrix} 0 \\ 0 \\ 0 \end{pmatrix}. \) 
\subsection{Find a \emph{p}-value for testing:}
\label{sec-1-8}

\(H_0 : \begin{pmatrix} 0 & 1 & -1 & 0 & 0 \\ 0 & 0 & 1 & -1 & 0
\end{pmatrix} \begin{pmatrix} \mu \\ \tau_1 \\ \tau_2 \\ \tau_3
\\ \tau_4 \end{pmatrix} = \begin{pmatrix} 10 \\ 0  \end{pmatrix}.\)
\section{Problem 2 In the following make use of the data in Problem 4 of Homework Assignment 3. Consider a regression of \emph{y} on $x_1, x_2,\ldots,x_5$. Use R matrix calculations to do the following in a full rank Gauss-Markov normal linear model.}
\label{sec-2}
\subsection{Find 90\% two-sided condifence limits for $\sigma$.}
\label{sec-2-1}
\subsection{Find 90\% two-sided condifence limits for the mean response under the conditions of data point \#1.}
\label{sec-2-2}
\subsection{Find 90\% two-sided condifence limits for the difference in mean responses under the conditions of data points \#1 and \#2. .}
\label{sec-2-3}
\subsection{Find a \emph{p}-value for testing the hypothesis that the conditions of data points \#1 and \#2 produce the same mean response.}
\label{sec-2-4}
\subsection{Find 90\% two-sided prediction limits for an additional response for the set of conditions \$x$_1$ = 0.005, x$_2$ = 0.45, x$_3$ = 7, x$_4$ = 45,\$ and $x_5 = 6$.}
\label{sec-2-5}
\subsection{Find a \emph{p}-value for testing the hypothesis that a model including only \emph{x$_1$}, \emph{x$_3$}, and \emph{x$_5$} is adequare for ``explaining'' home price.}
\label{sec-2-6}

(Hint: write it in the form of \emph{H$_0$} : \textbf{C$\beta$ = 0} ).
 The full model in this problem is \emph{y} = $\beta$$_0$ + $\beta$$_1$ x$_1$ +
 $\beta$$_2$ x$_2$ + $\beta$$_3$ x$_3$ + $\beta$$_4$ x$_4$ + $\beta$$_5$ x$_5$ + $\epsilon$. 
 The reduced model to test is \emph{H$_0$} : $\beta$$_2$ = $\beta$$_4$ = 0 or \emph{y} =
 $\beta$$_0$ + $\beta$$_1$ x$_1$ + $\beta$$_3$ x$_3$ + $\beta$$_5$ x$_5$ + $\epsilon$. This can be written \textbf{C$\beta$ = 0}, with C = (0 0 1 0 1 0).

We can create a \emph{p}-value to test these models using an F statistic,
constructed out of the ratio of the difference in regression sum of
squares between the full (SSR$_{\mathrm{full}}$) and reduced(SSR$_{\mathrm{reduced}}$) models and the sum of squared
error (SSE). These quantities are independent and follow a
non-central $\chi$$^2$(\emph{h},$\lambda$) and central $\chi$$^2$(\emph{n-k-1})
respectively where \emph{n} is the number of observations, \emph{k} is the
number of parameters in the full model, and \emph{h} is the difference in the
number of parameters between the full and reduced models. The
non-centrality parameter $\lambda$ can be written \textbf{$\beta$$_2$'[X$_2$'X$_2$ - X$_2$'X$_1$(X$_1$'X$_1$)$^{\mathrm{-1}}$X$_1$'X$_2$]$\beta$$_2$/2$\sigma$$^2$} where \textbf{X$_1$} and \textbf{X$_2$}
form a partition of \textbf{X} such that we can write: $$\mathbf{y} =
\mathbf{X\beta} + \mathbf{\epsilon} =
(\mathbf{X}_1,\mathbf{X}_2)\begin{pmatrix} \mathbf{\beta}_1
\\ \mathbf{\beta}_2 \end{pmatrix} + \mathbf{\epsilon} =
\mathbf{X}_1\mathbf{\beta}_1 + \mathbf{X}_2\mathbf{\beta}_2 +
\mathbf{\epsilon} $$

And the reduced model would be \textbf{y} = \textbf{X$_1$$\beta$$_1$$^{\star}$} + $\epsilon$$^{\star}$.



\begin{verbatim}
#Find SSR in the full model.
SSR_Bf <- t(bhat_B) %*% t(X_B) %*% Y_B - (length(Y_B)*(mean(Y_B))^2)

#create reduced model design matric and X1_B and estimator bhat1_B
X1_B <- X_B[,-c(3,5)]
bhat1_B <- ginv(t(X1_B)%*%X1_B) %*% t(X1_B) %*% Y_B
SSR_Br <- t(bhat1_B) %*% t(X1_B) %*% Y_B - (length(Y_B)*(mean(Y_B))^2)

SSE_B <- t(Y_B)%*%Y_B - t(bhat_B)%*%t(X_B)%*%Y_B

F_2f <- ((SSR_Bf - SSR_Br)/2)/(SSE_B/(length(Y_B) - qr(X_B)$rank))

pf_2f <- pf(F_2f, 2, (length(Y_B)-(qr(X_B)$rank)), lower.tail=F)
pf_2f
\end{verbatim}

This gives us a \emph{p}-value of \texttt{3.19090353910822e-13}.
\section{Problem 3}
\label{sec-3}
\subsection{In the context of Problem 1, part g), suppose that in fact $\tau$$_1$ = $\tau$$_2$, $\tau$$_3$ = $\tau$$_4$ = $\tau$$_1$ - d$\sigma$. What is the distribution of the F statistic?}
\label{sec-3-1}
\subsection{Use R to plot the power of the $\alpha$ = 0.05 level test as a function of \emph{d} for \emph{d} $\in$ [-5,5], that is plotting \emph{P} (F > the cut-off value) against \emph{d}. The R function pf(q,df1,df2,ncp) will compute cumulative (non-central) F probabilities for you corresponding to the value q, for degrees of freedom df1 and df2 when the noncentrality parameter is ncp.}
\label{sec-3-2}


\newpage
\section{Appendix: Tangled R code}
\label{sec-4}


\lstinputlisting{DabbishHW4a.R}

\end{document}
