% Created 2015-04-09 Thu 15:02
\documentclass[11pt]{article}
\usepackage[utf8]{inputenc}
\usepackage[T1]{fontenc}
\usepackage{fixltx2e}
\usepackage{graphicx}
\usepackage{longtable}
\usepackage{float}
\usepackage{wrapfig}
\usepackage{soul}
\usepackage{textcomp}
\usepackage{marvosym}
\usepackage{wasysym}
\usepackage{latexsym}
\usepackage{amssymb}
\usepackage{hyperref}
\tolerance=1000
\usepackage{methodshw, amsmath}
\providecommand{\alert}[1]{\textbf{#1}}

\title{8004 Homework 9}
\author{Nooreen Dabbish}
\date{\today}
\hypersetup{
  pdfkeywords={},
  pdfsubject={},
  pdfcreator={Emacs Org-mode version 7.9.3f}}

\begin{document}

\maketitle




\section{Data is generated from the exponetial distribution with density}
\label{sec-1}

$$f(y) = \lambda \exp (-\lambda y),\, \text{where}\,\,\lambda,y>0.$$
\subsection{Show that it belongs to the exponential family distributions be indentifyting $\theta$, b($\theta$), $\phi$, a($\phi$) and c(y;$\phi$).}
\label{sec-1-1}


An exponential family distribution can be written in the form
$$\exp\left\{ \frac{y\theta - b(\theta)}{a(\phi)} + c(y;\phi) \bigg\}.$$

We write: $$f(y|\lambda) = \exp(-\lambda y + \log \lambda)$$ and
equate $\theta$ = -$\lambda$, b($\theta$) = -$\log$ $\lambda$ =-$\log$(-$\theta$),
(note that $\lambda$ >0, so $\theta$ < 0), 
$\phi$ = 1, a($\phi$) = 1, c(y;$\phi$) = 1.
\subsection{What is the canonical link and variance functions for a GLM with the response following the exponential distribution?}
\label{sec-1-2}


The link function connects the linear predictor $\mu$ to the parameter
$\theta$ in the exponential family distribution definition above. To
find the canonical link, we want $\mu$ = E(Y) = b'($\theta$). we find the first moment of y:

$$EY = \int_0^\infty \lambda y e^{-\lamda y} dy = \frac{1}{\lambda}$$

$$\text{Note}\,\, b'(\theta) = \frac{-1}{\theta} = \frac{1}{\lambda} =
\mu\,\,\,\text{and}\,\,\,b''(\theta) = \frac{1}{\theta^2} =
\frac{1}{\lambda^2} = \mu^2$$

We write $\theta$ as a funcion of $\mu$ $$\theta(\mu) = -\frac{1}{\mu}$$
$$b'^{-1}(\cdot) = \text{negative inverse function.}$$

Since var(Y) = b''($\theta$)a($\phi$), and a($\phi$) = 1, var(Y) =
b''($\theta$)  = $\mu$$^2$.
\subsection{Is there any practical difficulty for using the canonical link in practice?}
\label{sec-1-3}


Especially in small samples, canoncial links have desirable
properties. However, they may not be the best fit for a model
(McCullagh and Nelder pg 32).

Note in this case that the exponential mean is restricted to
positive values. However our $\mu$ is a linear combination of
predictors. This does not guarantee a positive restriction on our
estimates of the mean. 
\subsection{Express the deviance as a function of y$_i$ and fitted mean $\mu$$_i$ (i = 1, \ldots, n).}
\label{sec-1-4}


We have scaled deviance given by 
$$\frac{D(y;\hat{\mu})}{\phi} = 2 \sum \frac{w_i}{\phi}
\{y_i(\widetilde{\theta_i} - \hat{\theta_i} )-
b(\widetilde{\theta_i}) + b(\hat{\theta_i})\}$$

with a($\phi$) = $\phi$/w, \( \tilde{\theta}=\theta(y) \) denoting the full
model (n parameter) estimate of $\theta$, and \( \hat{\theta} =
\theta(\hat{\mu}) \)
denoting the null model (one parameter) estimate of $\theta$.

Evaluating for b($\theta$) = - $\log$ (-$\theta$) and $\phi$ = 1, w$_i$ = 1
gives 
$$D(y;\hat{\mu}) = 2 \sum yi(\widetilde{\theta_i} -
\hat{\theta_i} ) + \log \left(\frac{\widetilde{\theta_i}}{\hat{\theta_i}\right)\}$$
\section{Appendix: Tangled R Code}
\label{sec-2}


\lstinputlisting{DabbishHW9.R} 

\end{document}
