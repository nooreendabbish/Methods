% Created 2015-04-09 Thu 17:58
\documentclass[11pt]{article}
\usepackage[utf8]{inputenc}
\usepackage[T1]{fontenc}
\usepackage{fixltx2e}
\usepackage{graphicx}
\usepackage{longtable}
\usepackage{float}
\usepackage{wrapfig}
\usepackage{soul}
\usepackage{textcomp}
\usepackage{marvosym}
\usepackage{wasysym}
\usepackage{latexsym}
\usepackage{amssymb}
\usepackage{hyperref}
\tolerance=1000
\usepackage{methodshw, amsmath}
\providecommand{\alert}[1]{\textbf{#1}}

\title{8004 Homework 9}
\author{Nooreen Dabbish}
\date{\today}
\hypersetup{
  pdfkeywords={},
  pdfsubject={},
  pdfcreator={Emacs Org-mode version 7.9.3f}}

\begin{document}

\maketitle




\section{Data is generated from the exponetial distribution with density}
\label{sec-1}

$$f(y) = \lambda \exp (-\lambda y),\, \text{where}\,\,\lambda,y>0.$$
\subsection{Show that it belongs to the exponential family distributions be indentifyting $\theta$, b($\theta$), $\phi$, a($\phi$) and c(y;$\phi$).}
\label{sec-1-1}


An exponential family distribution can be written in the form
$$\exp\left\{ \frac{y\theta - b(\theta)}{a(\phi)} + c(y;\phi) \bigg\}.$$

We write: $$f(y|\lambda) = \exp(-\lambda y + \log \lambda)$$ and
equate $\theta$ = -$\lambda$, b($\theta$) = -$\log$ $\lambda$ =-$\log$(-$\theta$),
(note that $\lambda$ >0, so $\theta$ < 0), 
$\phi$ = 1, a($\phi$) = 1, c(y;$\phi$) = 1.
\subsection{What is the canonical link and variance functions for a GLM with the response following the exponential distribution?}
\label{sec-1-2}


The link function connects the linear predictor $\mu$ to the parameter
$\theta$ in the exponential family distribution definition above. To
find the canonical link, we want $\mu$ = E(Y) = b'($\theta$). we find the first moment of y:

$$EY = \int_0^\infty \lambda y e^{-\lamda y} dy = \frac{1}{\lambda}$$

$$\text{Note}\,\, b'(\theta) = \frac{-1}{\theta} = \frac{1}{\lambda} =
\mu\,\,\,\text{and}\,\,\,b''(\theta) = \frac{1}{\theta^2} =
\frac{1}{\lambda^2} = \mu^2$$

We write $\theta$ as a funcion of $\mu$ $$\theta(\mu) = -\frac{1}{\mu}$$
$$b'^{-1}(\cdot) = \text{negative inverse function.}$$

Since var(Y) = b''($\theta$)a($\phi$), and a($\phi$) = 1, var(Y) =
b''($\theta$)  = $\mu$$^2$.
\subsection{Is there any practical difficulty for using the canonical link in practice?}
\label{sec-1-3}


Especially in small samples, canoncial links have desirable
properties. However, they may not be the best fit for a model
(McCullagh and Nelder pg 32).

Note in this case that the exponential mean is restricted to
positive values. However our $\mu$ is a linear combination of
predictors. This does not guarantee a positive restriction on our
estimates of the mean. 
\subsection{Express the deviance as a function of y$_i$ and fitted mean $\mu$$_i$ (i = 1, \ldots, n).}
\label{sec-1-4}


We have scaled deviance given by 
$$\frac{D(y;\hat{\mu})}{\phi} = 2 \sum \frac{w_i}{\phi}
\{y_i(\widetilde{\theta_i} - \hat{\theta_i} )-
b(\widetilde{\theta_i}) + b(\hat{\theta_i})\}$$

with a($\phi$) = $\phi$/w, \( \tilde{\theta}=\theta(y) \) denoting the full
model (n parameter) estimate of $\theta$, and \( \hat{\theta} =
\theta(\hat{\mu}) \)
denoting the null model (one parameter) estimate of $\theta$.

Evaluating for b($\theta$) = - $\log$ (- $\theta$) and $\phi$ = 1, w$_i$ = 1
gives 
$$D(y;\hat{\mu}) = 2 \sum \bigg( y_i(\widetilde{\theta_i} - 
\hat{\theta_i} ) + \log \left(\frac{\widetilde{\theta_i}}{\hat{\theta_i}}\right)\bigg)$$

From above, we have that \( \theta(\mu) = -\frac{1}{\mu} \).
Evaluating for \( \hat{\theta_i} = 1/\hat{\mu}_i) \) and \(
\tilde{\theta_i} = y_i \) gives
$$D(y;\hat{\mu}) = 2 \sum \bigg( y_i \left(-\frac{1}{y_i} + \frac{1}
{\hat{\mu_i}} \right) + \log \left(\frac{\hat{\mu}_i}
{y_i}}\right)\bigg)
= 2 \sum \bigg(\left(\frac{y_i - \hat{\mu_i}}
{\hat{\mu_i}} \right) + \log \left(\frac{\hat{\mu}_i}
{y_i}}\right)\bigg)$$
\section{Consider the Orings data of Faraway(2006) where the number of damaged ones out of six orings and corresponding temperatures of space shuttle launches are recorded.}
\label{sec-2}



\begin{verbatim}
library(faraway)
data(orings)
\end{verbatim}
\subsection{Construct the appropriate test statistic for testing the effect of the temperature. State the approriate null distribution and give the p-value.}
\label{sec-2-1}


(Reference Chapter 2 of Faraway 2006)

We will use the deviance as a log likelihood test to test for the 
effect of temp. The null model, with only an intercept, is nested 
in the model including temperature. The difference of deviance
measures D$_S$ - D$_L$ follows a $\chi$$^2$ distribution with degrees of
freedom equal to the number of parameters excluded from the nested
model, in this case  df=1 under the null hypothesis that the
additional parameters do not contribute significantly to the model.



We write:

\begin{verbatim}
logitmod <- glm(cbind(damage, 6-damage) ~temp, family=binomial, orings)
devtest <- pchisq(logitmod$null-logitmod$deviance,
                  logitmod$df.null-logitmod$df.residual,lower=FALSE)
\end{verbatim}
 
 And obtain a highly significant p-value of \texttt{2.747e-06}.
 
\subsection{Does it affect the conclusion by changing the link function to probit and other link functions.}
\label{sec-2-2}


The deviance function derivation in part 1d) illustrated that the
form of the link function will impact the form of the deviance
function. It therefore seems possible that changing the link function
could alter the conclusion. With such a highly significant p-value
for the effect of temperature, this seems unlikely. Let's see.
\subsubsection{Probit link function}
\label{sec-2-2-1}



\begin{verbatim}
probitmod <- glm(cbind(damage, 6-damage) ~temp, family=binomial(link=probit), orings)
probitdevtest <- pchisq(probitmod$null-probitmod$deviance,
                  probitmod$df.null-probitmod$df.residual,lower=FALSE)
\end{verbatim}

And agian obtain a highly significant p-value of \texttt{5.187e-06}.
\subsubsection{Complementary Log-Log link function}
\label{sec-2-2-2}



\begin{verbatim}
clogmod <- glm(cbind(damage, 6-damage) ~temp, family=binomial(link=cloglog), orings)
clogdevtest <- pchisq(clogmod$null-clogmod$deviance,
                  clogmod$df.null-clogmod$df.residual,lower=FALSE)
\end{verbatim}

And again obtain a highly significant p-value of \texttt{1.734e-06}.
\subsubsection{Cauchit link function}
\label{sec-2-2-3}



\begin{verbatim}
cauchitmod <- glm(cbind(damage, 6-damage) ~temp, family=binomial(link=cauchit), orings)
cauchitdevtest <- pchisq(cauchitmod$null-cauchitmod$deviance,
                  cauchitmod$df.null-cauchitmod$df.residual,lower=FALSE)
\end{verbatim}

And again obtain a highly significant p-value of
\texttt{3.174e-07}.
\subsubsection{Log link function}
\label{sec-2-2-4}



\begin{verbatim}
logmod <- glm(cbind(damage, 6-damage) ~temp, family=binomial(link=log), orings)
logdevtest <- pchisq(logmod$null-logmod$deviance,
                  logmod$df.null-logmod$df.residual,lower=FALSE)
\end{verbatim}

And again obtain a highly significant p-value of
\texttt{9.24e-07}.

While all of these p-values were significant, it is clear there is
some amount of variation. Moreover, this
may not be the case for a larger p-value or a different test run with
different link functions. Quite helpfully, R restricts the use of
link functions with binomial data to the five tested here. I infer
this is in order to guarantee a value for p between 0 and 1.
\subsection{Creating a new column of response as the indicator on whether or not there is some oring damage in that launch. Refit this binary reponse to a GLM with the logit link.}
\label{sec-2-3}




With the new model, the significance of the launch temperature is much
lower (but still significant) at \texttt{0.004804}.
\subsection{Which model do you prefer? The one in part (a) or part (c)? Why?}
\label{sec-2-4}


I prefer the model in part a, because the model in part (c) is
throwing away additional data. However, one advantage of the
indicator/binary model is that there is one data point at lower
temperature with a high proportion of damage which may be acting as
an outlier or influential point.
\section{Appendix: Tangled R Code}
\label{sec-3}


\lstinputlisting{DabbishHW9a.R} 

\end{document}
