% Created 2015-01-12 Mon 14:50
\documentclass[11pt]{article}
\usepackage[utf8]{inputenc}
\usepackage[T1]{fontenc}
\usepackage{fixltx2e}
\usepackage{graphicx}
\usepackage{longtable}
\usepackage{float}
\usepackage{wrapfig}
\usepackage{soul}
\usepackage{textcomp}
\usepackage{marvosym}
\usepackage{wasysym}
\usepackage{latexsym}
\usepackage{amssymb}
\usepackage{hyperref}
\tolerance=1000
\usepackage{amsmath}
\providecommand{\alert}[1]{\textbf{#1}}

\title{Homework 10 Redo}
\author{Nooreen Dabish}
\date{\today}
\hypersetup{
  pdfkeywords={},
  pdfsubject={},
  pdfcreator={Emacs Org-mode version 7.9.3f}}

\begin{document}

\maketitle

\section{Problem 1: US Crime Data}
\label{sec-1}
\subsection{Plot the scatterplot matrix between the variables.}
\label{sec-1-1}
\subsection{Construct a linear model to study the relationship between Crime(Y) and Prob, adjusting for the effect of the 13 char. variables.}
\label{sec-1-2}
\subsubsection{Description of the model:}
\label{sec-1-2-1}


The adjusted effect model is:
$$\mathbf{Y} = \beta_0 + \mathbf{X_1}\beta_1+\ldots+\mathbf{X_{p-1}}\beta_{p-1}+\mathbf{X_p}\beta_p+\epsilon$$

where:

\begin{equation*}
   \mathbf{Y} = 
   \begin{pmatrix}
    Y_1\\
    Y_2\\
    \vdots\\
     Y_n
  \end{pmatrix}\, ; \,
  \mathbf{X} =
  \begin{pmatrix}
    1      & X_{1,1} & \cdots & X_{1,p-1}\\
    1      & X_{2,1} & \cdots & X_{2,p-1}\\
    \vdots & \vdots  & \ddots  & \vdots \\
    1      & X_{n,1} & \cdots & X_{n,p-1}
  \end{pmatrix} \, ; \,
   \mathbf{\beta} = 
   \begin{pmatrix}
    \beta_0\\
    \beta_1\\
    \vdots\\
     \beta_{p-1}
  \end{pmatrix}\, ; \,
   \mathbf{\epsilon} = 
   \begin{pmatrix}
    \epsilon_1\\
    \epsilon_2\\
    \vdots\\
    \epsilon_n
  \end{pmatrix}
  \end{equation*}

 There are \texttt{15} covariates in the
 data set, so $p = 15$. There are \$ n =
 =47=\$ observations in the data set.  
Our model assumes that:
\begin{itemize}
\item $E(\mathbf{\epsilon}) = 0$
\item $\mathrm{Var}(\mathbf{\epsilon}) = \sigma^2\mathbf{I}$
\end{itemize}
 
 *** E
\subsubsection{Estimating the paramaters in the model:}
\label{sec-1-2-2}

 $$\hat{\mathbf{\beta}}_{LSE} =
 (\mathbf{X}^{\mathrm{T}}\mathbf{X})^{-1}\mathbf{X}^{\mathrm{T}}\mathbf{Y}$$

 and

 $$\hat{\sigma}^2_{LSE} =
\frac{\mathbf{Y}^{\mathrm{T}}\mathbf{Y} -
 \mathbf{Y}^{\mathrm{T}} \mathbf{P_X}\mathbf{Y}}{n - p}$$

and

$$\hat{\mathrm{Var}}(\hat{\mathbf{\beta}}_{LSE}) =
\hat{\sigma}^2(\mathbf{X}^{\mathrm{T}}\mathbf{X})^{-1}$$

where

The projection matrix $\mathbf{P_X}$ is $\mathbf{X}(\mathbf{X}^{\mathrm{T}}\mathbf{X})^{-1}\mathbf{X}^{\mathrm{T}}$.
\subsubsection{Test for the effect of Prob, the probability of imprisonment on Crime, adjusting for the other variables.}
\label{sec-1-2-3}

 Prob is determined as the ratio of the number of commitments to the
 number of offenses.
\begin{itemize}

\item Hypotheses:
\label{sec-1-2-3-1}%
\begin{itemize}
\item \textbf{Null} $H_0:\, \beta 14
\end{itemize}
 = 0$ 
\begin{itemize}
\item \textbf{Alternate} $H_A:\, \beta 14 \neq 0$
\end{itemize}

\begin{align*} 
   \hat{\mathbf{\beta}}_{LSE} &\sim N(\mathbf{\beta}, \sigma^2(\mathbf{X}^{\mathrm{T}}\mathbf{X})^{-1}) \\
   q^\mathrm{T}\hat{\mathbf{\beta}}_{LSE} &\sim N(q^\mathrm{T}\mathbf{\beta},\sigma^2q^\mathrm{T}(\mathbf{X}^{\mathrm{T}}\mathbf{X})^{-1}q)   \\
\frac{q^{\mathrm{T}}\hat{\mathbf{\beta}}_{LSE} - q^{\mathrm{T}}\mathbf{\beta}}{ \hat{\sigma}\sqrt{q^{\mathrm{T}}(\mathbf{X}^{\mathrm{T}}\mathbf{X})^{-1}q}} &\sim T_{n-p}
\text{Under $H_0$:}\\
\frac{q^{\mathrm{T}}\hat{\mathbf{\beta}}_{LSE}}{ \hat{\sigma}\sqrt{q^{\mathrm{T}}(\mathbf{X}^{\mathrm{T}}\mathbf{X})^{-1}q}} &\sim T_{n-p}
\end{align*}
 

\item Rejection criterion\\
\label{sec-1-2-3-2}%
$$\left\lvert \frac{q^{\mathrm{T}}\hat{\mathbf{\beta}}_{LSE}}{
\hat{\sigma}\sqrt{q^{\mathrm{T}}(\mathbf{X}^{\mathrm{T}}\mathbf{X})^{-1}q}}
\right\rvert > t_{n-p}^{-1} (1 - \alpha/2)$$ 




\begin{center}
\begin{tabular}{rrrrrrrrrrrrrr}
 15.1  &  1  &   9.1  &   5.8  &   5.6  &   0.51  &     95  &   33  &  30.1  &  0.108  &  4.1  &  3940  &  26.1  &  0.084602  \\
 14.3  &  0  &  11.3  &  10.3  &   9.5  &  0.583  &  101.2  &   13  &  10.2  &  0.096  &  3.6  &  5570  &  19.4  &  0.029599  \\
 14.2  &  1  &   8.9  &   4.5  &   4.4  &  0.533  &   96.9  &   18  &  21.9  &  0.094  &  3.3  &  3180  &    25  &  0.083401  \\
 13.6  &  0  &  12.1  &  14.9  &  14.1  &  0.577  &   99.4  &  157  &     8  &  0.102  &  3.9  &  6730  &  16.7  &  0.015801  \\
 14.1  &  0  &  12.1  &  10.9  &  10.1  &  0.591  &   98.5  &   18  &     3  &  0.091  &    2  &  5780  &  17.4  &  0.041399  \\
 12.1  &  0  &    11  &  11.8  &  11.5  &  0.547  &   96.4  &   25  &   4.4  &  0.084  &  2.9  &  6890  &  12.6  &  0.034201  \\
\end{tabular}
\end{center}





\item The $(1-\alpha)$ Confidence Interval of $q^{\mathrm{T}}\beta$\\
\label{sec-1-2-3-3}%
$$q^{\mathrm{T}}\hat{\mathbf{\beta}} \pm
t^{-1}_{n-p}(1-\alpha/2)s\sqrt{q^{\mathrm{T}}(\mathbf{X}^{\mathrm{T}}\mathbf{X})^{-1}q}$$


\end{itemize} % ends low level

\end{document}
