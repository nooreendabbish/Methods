\documentclass[12pt]{article}

\usepackage{amscd}
\usepackage{fullpage}
\usepackage{tabularx}
\usepackage{amsmath,mathtools,amssymb,amsfonts, amsthm}
\usepackage{caption}
\usepackage{graphicx,graphics,psfrag,epsf,subcaption,epsfig}
\usepackage{setspace,parskip}
\usepackage{latexsym}
\usepackage{indentfirst}
\usepackage{hyperref}
\usepackage{rotating}
\usepackage{multirow}
\usepackage{framed,color}
%\usepackage{xcolor}
\usepackage{listings,bera,pstricks-add,caption,etoolbox}


\definecolor{shadecolor}{rgb}{1,0.8,0.3}

%\usepackage{showkeys}
\newcommand{\blind}{0}
\newcommand{\ignore}[1]i{}
\newcommand{\blank}{\vspace{3ex}}
%\usepackage{amsrefs}
%\doublespacing
%\fontsize{12pt}{\baselineskip}
%\parskip=\baselineskip %\advance\parskip by 0pt plus 2pt
\setlength\parskip{10pt}

\lstset{  %
    breaklines=true,%default : false 
    breakindent=10pt,%default: 20pt 
    linewidth=\linewidth,%default : \linewidth,
    basicstyle=\ttfamily,% cannot take arguments
    %keywordstyle=\color{Blue}\sffamily\bfseries,                                
    %identifierstyle=\color{Black},                                      
    %commentstyle=\color{OliveGreen}\itshape,                                    
    stringstyle=\rmfamily,                                                      
    showspaces=false,%default false
    showstringspaces=false,%default: true
    backgroundcolor=\color{shadecolor},
    frame=single,%default frame=none 
    rulecolor=\color{black},  
    % the following must be defined to make hacking work.
    framerule=0.4pt,%expand outward 
    framesep=3pt,%expand outward
    xleftmargin=3.4pt,%to make the frame fits in the text area. 
    xrightmargin=3.4pt,%to make the frame fits in the text area. 
    tabsize=2,%default :8 only influence the lstlisting and lstinline.
    aboveskip=7pt plus 2pt, 
    belowskip=-8pt plus 2pt
} 


\newtheorem{theorem}{Theorem}
\newtheorem{corollary}{Corollary}
\newtheorem{proposition}{Proposition}
\newtheorem{lemma}{Lemma}
\newtheorem{example}{Example}
\newtheorem{definition}{Definition}

\newcommand{\thmref}[1]{Theorem~\ref{#1}}
\newcommand{\propref}[1]{Proposition~\ref{#1}}
%\newcommand{\corref}[1]{Corollary~\ref{#1}}
\newcommand{\lemref}[1]{Lemma~\ref{#1}}
\newcommand{\secref}[1]{\S\ref{#1}}
\newcommand{\figref}[1]{Figure~\ref{#1}}

\newcommand\independent{\protect\mathpalette{\protect\independenT}{\perp}}
\def\independenT#1#2{\mathrel{\rlap{$#1#2$}\mkern2mu{#1#2}}}

\newcommand{\etc}{\textit{etc}.}
\newcommand{\etal}{\textit{et al}}
\newcommand{\ie}{\textit{i.e.}}
\newcommand{\eg}{\textit{e.g.}}
\newcommand{\st}{\textit{s.t.}}
\newcommand{\vs}{\textit{vs.}}
\newcommand{\iid}{\textit{i.i.d.}}

\newcommand{\Cov}{\mathrm{Cov}}
\newcommand{\Var}{\mathrm{Var}}
\newcommand{\Rank}{\mathrm{Rank}}
\newcommand{\trans}{\mathrm{T}}
\newcommand{\ud}{\,\mathrm{d}}
\newcommand{\uH}{\,\mathrm{H}}

\newcommand{\EE}{\,\mathbb{E}\,}
\newcommand{\PP}{\,\mathbb{P}\,}
\newcommand{\VV}{\,\mathbb{V}\,}
\newcommand{\RR}{\mathbb{R}}
\newcommand{\NN}{\mathbb{N}}
\newcommand{\Or}{\mathcal{O}}
\newcommand{\Ir}{\mathcal{I}}
\newcommand{\Br}{\mathcal{B}}
\newcommand{\Cr}{\mathcal{C}}
\newcommand{\Dr}{\mathcal{D}}
\newcommand{\Sr}{\mathcal{S}}
\newcommand{\Ur}{\mathcal{U}}
\newcommand{\Vr}{\mathcal{V}}
\newcommand{\Cf}{\mathfrak{C}}
\newcommand{\Sf}{\mathfrak{S}}

\newcommand{\va}{\mathbf{a}}
\newcommand{\vA}{\mathbf{A}}
\newcommand{\vb}{\mathbf{b}}
\newcommand{\vB}{\mathbf{B}}
\newcommand{\vc}{\mathbf{c}}
\newcommand{\vd}{\mathbf{d}}
\newcommand{\ve}{\mathbf{e}}
\newcommand{\vh}{\mathbf{h}}
\newcommand{\vi}{\mathbf{i}}
\newcommand{\vI}{\mathbf{I}} % this is capital ai
\newcommand{\vJ}{\mathbf{J}}
\newcommand{\vK}{\mathbf{K}}
\newcommand{\vk}{\mathbf{k}}
\newcommand{\vl}{\mathbf{l}} % this is small el
\newcommand{\vL}{\mathbf{L}}
\newcommand{\vm}{\mathbf{m}} 
\newcommand{\vn}{\mathbf{n}}
\newcommand{\vp}{\mathbf{p}}
\newcommand{\vP}{\mathbf{P}}
\newcommand{\vq}{\mathbf{q}}
\newcommand{\vr}{\mathbf{r}}
\newcommand{\vu}{\mathbf{u}}
\newcommand{\vR}{\mathbf{R}}
\newcommand{\vS}{\mathbf{S}}
\newcommand{\vT}{\mathbf{T}}
\newcommand{\vt}{\mathbf{t}}
\newcommand{\vx}{\mathbf{x}}
\newcommand{\vX}{\mathbf{X}}
\newcommand{\vy}{\mathbf{y}}
\newcommand{\vY}{\mathbf{Y}}
\newcommand{\vone}{\boldsymbol{1}}
\newcommand{\vzero}{\boldsymbol{0}}
\newcommand{\balp}{\boldsymbol\alpha}
\newcommand{\bbeta}{\boldsymbol\beta}
\newcommand{\beps}{\boldsymbol\epsilon}
\newcommand{\beeta}{\boldsymbol\eta}
\newcommand{\bdel}{\boldsymbol\delta}
\newcommand{\bmu}{\boldsymbol\mu}
\newcommand{\bth}{\boldsymbol\theta}
\newcommand{\bvth}{\boldsymbol\vartheta}
\newcommand{\bphi}{\boldsymbol{\phi}}
\newcommand{\bLam}{\boldsymbol\Lambda}
\newcommand{\bSigma}{\boldsymbol\Sigma}
\newcommand{\bpi}{\boldsymbol\pi}

\begin{document}

\title{\textsc{FALL 2013\\
Stat 8003: Statistical Methods I\\
Lecture 1}}
\author{Jichun Xie}
\date{}

\maketitle

\section{Syllabus}

\section{Examples of Statistics and Its Applications}

\subsection{Investigation of Salary Discrimination}

When a group of workers believes that their employer is illegally
discriminating against the group, legal remedies are often
available. Usually such groups are minorities consisting of a racial,
ethnic, gender, or age group. The discrimination may deal with salary,
benefits, working conditions, mandatory retirement, etc. The
statistical evidence is often crucial to the development of the legal
case.

For example, if there is doubt about salary discrimination between
male and female workers. What would statisticians do? Usually, they
collect data from a subpoena by the legal team. The variables include
salaries, years of experience, years of education, a measure of
current job responsibility or complexity, a measure of the worker's
current productivity, and, the last but not the least, gender. The
statisticians consider linear regression model. First they put all the
variables other than gender in the model, and then they put all the
variables in the model. If the second model works much better for the
data, or equivalently, gender is proven to be statistically
significant, then this may be regarded as statistical evidence of
salary discrimination between female and male employers. We will
discuss in detail how to analyze data using linear regression model
and how to judge whether a variable is statistically significant in
this class.

\subsection{Detection of Academic Fabrication}

Five years ago, Duke University announced it had found the holy grail
of cancer research. Dr.~Anil Potti in the Dr.~Joseph Nevins group
discovered how to match a patient's tumor to the best chemotherapy
drug. It was a breakthrough because every person's DNA is unique, so
every tumor is different. A drug that kills a tumor in one person, for
example, might not work in another. The research was published in the
most prestigious medical journals. Duke is excited as well as the 112
patients who signed up for the revolutionary therapy.  Doctors
everywhere were eager to save lives with the new discovery. Later,
however, two statisticians at MD Anderson Cancer Center began
analyzing Dr.~Potti's data to verify his results. However, they
noticed that something really odd that they couldn't explain. They
then emailed their questions to Duke. Dr.~Potti admitted a few
clerical errors, but he said that the new work confirmed his
results. And Duke moved ahead. Dr.~Nevins and Dr.~Potti started a
company to market the process. They made a fortune. Patients enrolled
in the clinical trials are assigned with the treatment they would
believe to be the best for them. However, at MD Anderson Cancer
Center, the statisticians kept finding errors that they thought were
alarming.  The statisticians then wrote a statistical paper analyzing
the errors they found in the revolutionary treatment. And they
submitted the paper to Annals of Applied Statistics. They also
contacted Duke, and Duke invited some external review committee to
analyze Dr.~Potti's investigation. After three months, the review
committee concluded that Dr.~Potti was not wrong.  So the clinical
trial went on. Things haven't been changed too much until later, the
editor of a small independent newsletter, called ``The Cancer Lette'',
got a tip from a confidential source: check Dr.~Potti's Rhodes
scholarship. Dr.~Potti claimed he got the scholarship when he applied
for federal grants. The trouble is that it wasn't true. Till then,
Dr.~Nevins realized that maybe Dr.~Potti is faking the data. He then
reviewed the original data and unfortunately his doubt has been
confirmed. The data has been manipulated, and lots of the people, the
patients, Duke including himself, have been deceived. It turned out
that the therapy doesn't work at all. Their theory is wrong. But some
of the patients have already died.  Well, there were statistical
evidences that the data might be manipulated when the clinical trials
just started. However, the evidence was neglected or ignored. If these
evidences could be treated with enough attention, maybe the fraud
could be discovered earlier and fewer patients would die.


\subsection{Statistical Tests for Drugs}

\textbf{Abuse of Diethylstilbestrol (DES)}

Wikipedia: \url{http://en.wikipedia.org/wiki/Diethylstilbestrol}

Diethylstilbestrol (DES, former BAN stilboestrol) is a synthetic
nonsteroidal estrogen that was first synthesized in 1938. It is also
classified as an endocrine disruptor. Human exposure to DES occurred
through diverse sources, such as dietary ingestion from supplemented
cattle feed and medical treatment for certain conditions, including
breast and prostate cancers. From about 1940 to 1970, DES was given to
pregnant women in the mistaken belief it would reduce the risk of
pregnancy complications and losses. In 1971, DES was shown to cause a
rare vaginal tumor in girls and women who had been exposed to this
drug in utero. The United States Food and Drug Administration
subsequently withdrew DES from use in pregnant women. Follow-up
studies have indicated DES also has the potential to cause a variety
of significant adverse medical complications during the lifetimes of
those exposed. The United States National Cancer Institute
recommends women born to mothers who took DES undergo special
medical exams on a regular basis to screen for complications as a
result of the drug. Individuals who were exposed to DES during their
mothers' pregnancies are commonly referred to as "DES daughters" and
"DES sons".


\subsection{Statistical Learning and Data Mining}

\textbf{Handwritten digit recognition}

 \begin{itemize}
  \item Goal: identify single digits 0 -- 9 based on images.
  \item Raw data: images that are scaled segments from five digit ZIP codes.
\begin{itemize}
\item 16 $\times$ 16 eight-bit grayscale maps 
\item Pixel intensities range from 0
  (black) to 255 (white).
\end{itemize}
\item Input data: a 256 dimension vector, or feature vectors with lower dimensions.
  \end{itemize}

      \begin{figure}[h]
    \centering
     \includegraphics[width=0.8\textwidth, natwidth=343, natheight=224]{hand.pdf}
    \caption{Elements of Statistical Learning
        \textcopyright   Hastie, Tibshirani \& Friendman 2001
        \hspace{3ex}  Chapter 1}
  \end{figure}

\textbf{Foreground motion detection}
 \begin{itemize}
\item Goal: extract moving objects from a video sequence.
\item Raw data: grayscale image sequence represented by matrices of size $m
  \times n\times t$, or color image sequence represented by 3 such arrays.
\item Videos: 
  \begin{itemize}
  \item \url{http://www.youtube.com/watch?v=7pE-4eSMUs4}
  \item \url{http://www.youtube.com/watch?v=F6rxhlkgJkk}
  \item \url{http://www.youtube.com/watch?v=yplmDhOgNM8}
  \end{itemize}
  \end{itemize}


\section{Introduction to LaTeX}

References: 
\begin{itemize}
\item LaTeX-project: \url{http://www.latex-project.org/}
\item LaTeX wikibook: \url{http://en.wikibooks.org/wiki/LaTeX}
\end{itemize}

\subsection{Basics}

\subsubsection{Global Structures}
\begin{lstlisting}
\documentclass{...}  
\usepackage{...}

\begin{document}
...
\end{document}

\end{lstlisting}


\begin{itemize}
\item The preamble: the area between {\verb \documentclass{...} } and 
  {\verb \begin{document} }.
\item document text: the area between   {\verb \begin{document} } and
    {\verb \end{doucment} }.
\end{itemize}

\subsubsection{Text and Paragraph Formatting}

\begin{lstlisting}
  \emph{emphasis}, \textbf{bold}, \textit{italic}
\end{lstlisting}
  \emph{emphasis}, \textbf{bold}, \textit{italic}

\blank

\begin{lstlisting}
 \begin{verbatim}
  The verbatim environment
    simply reproduces every
   character you input,
  including all  s p a c e s!
 \end{verbatim}
\end{lstlisting}
\begin{verbatim}
The verbatim environment
  simply reproduces every
 character you input,
including all  s p a c e s!
\end{verbatim}  

\blank


\begin{lstlisting}
  \begin{doublespace}
    This paragraph has \\ double \\ line spacing.
  \end{doublespace}
\end{lstlisting}
  \begin{doublespace}
    This paragraph has \\ double \\ line spacing.
  \end{doublespace}

\blank

\begin{lstlisting}
  \begin{spacing}{2.5}
    This paragraph has \\ huge gaps \\ between lines.
  \end{spacing}
\end{lstlisting}
  \begin{spacing}{2.5}
    This paragraph has \\ huge gaps \\ between lines.
  \end{spacing}

\subsubsection{Labels and Cross Referencing}

Another good point of LaTeX is that you can easily reference almost
anything that is numbered (sections, figures, formulas), and LaTeX
will take care of numbering, updating it whenever necessary. The
commands to be used do not depend on what you are referencing, and
they are:

\begin{lstlisting}
  \label{marker}
\end{lstlisting}
You give the object you want to reference a marker, you can see it
like a name.

\begin{lstlisting}
  \ref{marker}
\end{lstlisting}
You can reference the object you have marked before. This prints the
number that was assigned to the object.

\begin{lstlisting}
  \pageref{marker}
\end{lstlisting}
It will print the number of the page where the object is.

\subsection{Mathematics}

One of the greatest motivating forces for Donald Knuth when he began
developing the original TeX system was to create something that
allowed simple construction of mathematical formulas, whilst looking
professional when printed. The fact that he succeeded was most
probably why TeX (and later on, LaTeX) became so popular within the
scientific community. Typesetting mathematics is one of LaTeX's
greatest strengths. It is also a large topic due to the existence of
so much mathematical notation.

If your document requires only a few simple mathematical formulas,
plain LaTeX has most of the tools that you will need. If you are
writing a scientific document that contains numerous complicated
formulas, you might need the \texttt{amsmath} package and the
\texttt{mathtools} packages. Include
\begin{lstlisting}
  \usepackage{amsmath,mathtools}
\end{lstlisting}
in the preamble.

\subsubsection{Mathematics Environments}

Text: 
\begin{lstlisting}
x in the math text mode: $x$, \(x\)
\end{lstlisting}
x in the math text mode: $x$, \(x\)

Displayed:
\begin{lstlisting}
There are Three ways to display a math equation:
$$ x + y = z $$
\[x + y = z\]
\begin{equation*}
  x + y = z
\end{equation*}
\end{lstlisting}
There are Three ways to display a math equation:
$$ x + y = z $$
\[x + y = z\]
\begin{equation*}
  x + y = z
\end{equation*}

Displayed with equation number:
\begin{lstlisting}
\begin{equation}\label{eq:xyz}
  x + y = z
\end{equation}
\end{lstlisting}
\begin{equation}\label{eq:xyz}
  x + y = z
\end{equation}
\begin{lstlisting}
Equation (\ref{eq:xyz}) on page \pageref{eq:xyz} describes a relationship between $x$, $y$ and $z$.
\end{lstlisting}
Equation (\ref{eq:xyz}) on page \pageref{eq:xyz} describes a relationship between $x$, $y$ and
$z$.

Multi-equation display:
\begin{lstlisting}
\begin{align*}
    x + y +z & =  6\\
    x - y & = 2\\
    x - z & = 4
\end{align*}
\end{lstlisting}
  \begin{align*}
    x + y +z & =  6\\
    x - y & = 2\\
    x - z & = 4
  \end{align*}



\subsubsection{Symbols}

Mathematics has many symbols! One of the most difficult aspects of learning LaTeX is remembering how to produce symbols. There are of course a set of symbols that can be accessed directly from the keyboard:
\begin{lstlisting}
 + - = ! / ( ) [ ] < > | ' :
\end{lstlisting}

Beyond those listed above, distinct commands must be issued in order
to display the desired symbols. There are a great deal of examples
such as Greek letters, set and relations symbols, arrows, binary
operators, etc. For example,
\begin{lstlisting}
   \forall x \in X, \quad \exists y \leq \epsilon
\end{lstlisting}
\[\forall x \in X, \quad \exists y \leq \epsilon\]

Fortunately, there's a tool that can greatly simplify the search for
the command for a specific symbol. Search \texttt{Detexify} and you
will find a website that allows you search for the command by inputing
handwriting symbols. There are also apps for iPhone and Android
phones. Some softwares such as WinEdt incorporates a list of basic
symbols in the toolbox. Another option would be to look in the ``The
Comprehensive LaTeX Symbol List" available at:

\centerline{\url{http://www.ctan.org/tex-archive/info/symbols/comprehensive}}

Now we introduce some basic symbols that are commonly used.

\textbf{Greek letters.} Greek letters are commonly used in
mathematics, and they are very easy to type in \emph{math mode}. You
just have to type the name of the letter after a backslash: if the
first letter is lowercase, you will get a lowercase Greek letter, if
the first letter is uppercase (and only the first letter), then you
will get an uppercase letter.
\begin{lstlisting}
  \alpha, \beta, \gamma, \Gamma, \pi, \Pi, \phi, \varphi, \Phi
\end{lstlisting}
  \[\alpha, \beta, \gamma, \Gamma,
  \pi, \Pi, \phi, \varphi, \Phi\]

\textbf{Operators.} An operator is a function that is written as a
  word: e.g. trigonometric functions (sin, cos, tan), logarithms and
  exponentials (log, exp). LaTeX has many of these defined as
  commands:
  \begin{lstlisting}
    \cos (2\theta) = \cos^2 \theta - \sin^2 \theta
  \end{lstlisting}
\[\cos (2\theta) = \cos^2 \theta - \sin^2 \theta\]
For certain operators such as limits, the subscript is placed
underneath the operator:
\begin{lstlisting}
  \lim_{x \to \infty} \exp(-x) = 0
\end{lstlisting}
  \[\lim_{x \to \infty} \exp(-x) = 0\]
For the modular operator there are two commands: {\verb \bmod } and
{\verb \pmod }
\begin{lstlisting}
  a \bmod b, \quad x \equiv a \pmod b
\end{lstlisting}
  \[a \bmod b, \quad x \equiv a \pmod b\]

  \textbf{Powers and indices.} Powers and indices are equivalent to
  superscripts and subscripts in normal text mode. The caret 
({\verb ^ }) character is used to raise something, and the underscore 
({\verb _ }) is for lowering. If more than one expression is raised or
 lowered, they should be grouped using curly braces (\{ and \}).
  \begin{lstlisting}
    k_{n+1} = n^2 + k_n^2 - k_{n-1}
  \end{lstlisting}
  \[k_{n+1} = n^2 + k_n^2 - k_{n-1}\] An underscore ({\verb _ }) can
  be used with a vertical bar ({\verb | }) to denote evaluation using
  subscript notation in mathematics:
  \begin{lstlisting}
  f(n) = n^5 + 4n^2 + 2 |_{n=17}
\end{lstlisting}
  \[f(n) = n^5 + 4n^2 + 2 |_{n=17}\]

  \textbf{Fractions and binomials.} A fraction is created using the
  {\verb \frac{numerator}{denominator} } command. (For those who need
  their memories refreshed, that's the top and bottom
  respectively!). Likewise, the binomial coefficient (aka the Choose
  function) may be written using the {\verb \binom } command:
  \begin{lstlisting}
  \frac{n!}{k!(n-k)!} = \binom{n}{k}
  \end{lstlisting}
  \[\frac{n!}{k!(n-k)!} = \binom{n}{k}\]
  For relatively simple fractions, it may be more aesthetically
  pleasing to use powers and indices: 
  \begin{lstlisting}
  ^3/_7, \quad 3/7
  \end{lstlisting}
\[  ^3/_7, \quad 3/7\]

\textbf{Roots.} The {\verb \sqrt } command creates a square root
surrounding an expression. It accepts an optional argument specified
in square brackets ([ and ]) to change magnitude:
\begin{lstlisting}
  \sqrt{\frac{a}{b}}, \quad \sqrt[n]{1+x+x^2+x^3+\ldots}
\end{lstlisting}
\[  \sqrt{\frac{a}{b}}, \quad \sqrt[n]{1+x+x^2+x^3+\ldots}\]

\textbf{Sums and integrals.} The {\verb \sum } and {\verb \int }
commands insert the sum and integral symbols respectively, with limits
specified using the caret ({\verb ^ }) and underscore 
({\verb _ }). The typical notation for sums is:
\begin{lstlisting}
  \sum_{i=1}^{10} t_i
\end{lstlisting}
\[\sum_{i=1}^{10} t_i\] 
The limits for the integrals follow the
same notation. It's also important to represent the integration
variables with an upright d, which in math mode is obtained through
the {\verb \mathrm{} } command, and with a small space separating it from the
integrand, which is attained with the {\verb \, } command.
\begin{lstlisting}
  \int_0^\infty \mathrm{e}^{-x}\,\mathrm{d}x
\end{lstlisting}
\[\int_0^\infty
\mathrm{e}^{-x}\,\mathrm{d}x\] 

\textbf{Automatic sizing.} Very often mathematical features will
differ in size, in which case the delimiters surrounding the
expression should vary accordingly. This can be done automatically
using the {\verb \left,\right }, and {\verb \middle } commands. Any
of the previous delimiters may be used in combination with these:
\begin{lstlisting}
  \left(\frac{x^2}{y^3}\right)
\end{lstlisting}
\[ \left(\frac{x^2}{y^3}\right)\]
\begin{lstlisting}
  P\left(A=2\middle|\frac{A^2}{B}>4\right)
\end{lstlisting}
\[  P\left(A=2\middle|\frac{A^2}{B}>4\right) \]
Curly braces are defined differently by using {\verb \left\{ } and
  {\verb \right\} },
\begin{lstlisting}
  \left\{\frac{x^2}{y^3}\right\}
\end{lstlisting}
\[ \left\{\frac{x^2}{y^3}\right\}\] 
If a delimiter on only one side of
an expression is required, then an invisible delimiter on the other
side may be denoted using a period ({\verb . }).
\begin{lstlisting}
  \left.\frac{x^3}{3}\right|_0^1
\end{lstlisting}
\[\left.\frac{x^3}{3}\right|_0^1\]

\textbf{Matrices and arrays.}
A basic matrix may be created using the matrix environment[3]: in
common with other table-like structures, entries are specified by row,
with columns separated using an ampersand ({\verb & }) and a new rows
separated with a double backslash ({\verb \\ })
\begin{lstlisting}
  \begin{matrix}
    a & b & c \\
    d & e & f \\
    g & h & i
  \end{matrix}
\end{lstlisting}
\[
\begin{matrix}
  a & b & c \\
  d & e & f \\
  g & h & i
\end{matrix}\] When writing down arbitrary sized matrices, it is
common to use horizontal, vertical and diagonal triplets of dots
(known as ellipses) to fill in certain columns and rows. These can be
specified using the {\verb \cdots,\vdots } and {\verb \ddots }
respectively:
\begin{lstlisting}
  A_{m,n} =
  \begin{pmatrix}
    a_{1,1} & a_{1,2} & \cdots & a_{1,n} \\
    a_{2,1} & a_{2,2} & \cdots & a_{2,n} \\
    \vdots  & \vdots  & \ddots & \vdots  \\
    a_{m,1} & a_{m,2} & \cdots & a_{m,n}
  \end{pmatrix}
\end{lstlisting}
\[A_{m,n} =
 \begin{pmatrix}
  a_{1,1} & a_{1,2} & \cdots & a_{1,n} \\
  a_{2,1} & a_{2,2} & \cdots & a_{2,n} \\
  \vdots  & \vdots  & \ddots & \vdots  \\
  a_{m,1} & a_{m,2} & \cdots & a_{m,n}
 \end{pmatrix}\]

\subsection{Tables}

\subsubsection{The Tabular Environment}
The tabular environment can be used to typeset tables with optional
horizontal and vertical lines. LaTeX determines the width of the
columns automatically.  The first line of the environment has the
form:
\begin{lstlisting}
  \begin{tabular}[pos]{table spec}
\end{lstlisting}
The \emph{table spec}  argument tells LaTeX the alignment to be used in 
each column and the vertical lines to insert.
The number of columns does not need to be specified as it is inferred
by looking at the number of arguments provided. It is also possible to add
vertical lines between the columns here. The following symbols are
available to describe the table columns (some of them require that the 
package array has been loaded):

\begin{tabular}[h]{|l|p{14cm}|}
  \hline
  l &	left-justified column\\
  c&	centered column\\
  r&	right-justified column\\
  p{'width'}&   paragraph column with text vertically aligned at the
  top\\
  m{'width'}& paragraph column with text vertically aligned in the middle (requires array package)\\
  b{'width'}&	paragraph column with text vertically aligned at the bottom (requires array package)\\
  |	&vertical line\\
  ||	&double vertical line\\
  \hline
\end{tabular}

The optional parameter pos can be used to specify the vertical
position of the table relative to the baseline of the surrounding
text. In most cases, you will not need this option. It becomes
relevant only if your table is not in a paragraph of its own. You can
use the following letters:

\begin{tabular}[h]{|l|l|}
\hline
b &	bottom \\
c & 	center (default)\\
p	& top\\
\hline
\end{tabular}

In the first line you have pointed out how many columns you want, 
their alignment and the vertical lines to separate them. Once in the 
environment, you have to introduce the text you want, separating 
between cells and introducing new lines. The commands you have 
to use are the following:

\begin{tabular}[h]{|l|p{14cm}|}
\hline
{\verb & } &	column separator\\
{\verb \\ } &  start new row (additional space may be specified after {\verb \\ }  using square brackets, such as {\verb \\[6pt] })\\
{\verb \hline } &	horizontal line\\
\hline
\end{tabular}

\subsubsection{Basic Examples}

This example shows how to create a simple table in LaTeX. 
It is a three-by-three table, but without any lines.
\begin{lstlisting}
\begin{tabular}{ l c r }
  1 & 2 & 3 \\
  4 & 5 & 6 \\
  7 & 8 & 9 \\
\end{tabular}
\end{lstlisting}

\begin{tabular}{ l c r }
  1 & 2 & 3 \\
  4 & 5 & 6 \\
  7 & 8 & 9 \\
\end{tabular}

Expanding upon that by including some vertical lines:
\begin{lstlisting}
\begin{tabular}{ l | c || r }
  1 & 2 & 3 \\
  4 & 5 & 6 \\
  7 & 8 & 9 \\
\end{tabular}
\end{lstlisting}

\begin{tabular}{ l | c || r }
  1 & 2 & 3 \\
  4 & 5 & 6 \\
  7 & 8 & 9 \\
\end{tabular}

To add horizontal lines to the very top and bottom edges of the table:
\begin{lstlisting}
\begin{tabular}{ l | c || r }
  \hline                        
  1 & 2 & 3 \\
  4 & 5 & 6 \\
  7 & 8 & 9 \\
  \hline  
\end{tabular}
\end{lstlisting}

\begin{tabular}{ l | c || r }
  \hline                        
  1 & 2 & 3 \\
  4 & 5 & 6 \\
  7 & 8 & 9 \\
  \hline  
\end{tabular}

And finally, to add lines between all rows, as well as 
centering (notice the use of the center environment --  of course, 
the result of this is not obvious from the preview on this web page):
\begin{lstlisting}
\begin{center}
  \begin{tabular}{ l | c || r }
    \hline
    1 & 2 & 3 \\ \hline
    4 & 5 & 6 \\ \hline
    7 & 8 & 9 \\
    \hline
  \end{tabular}
\end{center}
\end{lstlisting}

\begin{center}
  \begin{tabular}{ l | c || r }
    \hline
    1 & 2 & 3 \\ \hline
    4 & 5 & 6 \\ \hline
    7 & 8 & 9 \\
    \hline
  \end{tabular}
\end{center}

\subsubsection{Floating with Table}
To tell LaTeX we want to use our table as a float, we need to 
place a tabular environment in a table environment, which is able 
to float and add a label and caption.

The table environment is a type of floats just as figure is. In fact, 
they bear a lot of similarities (positionning, captions, \etc). 
\begin{lstlisting}
\begin{table}[position specifier]
  \centering
  \begin{tabular}{table spec}
    ... your table ...
  \end{tabular}
  \caption{This table shows some data}
  \label{tab:myfirsttable}
\end{table}
\end{lstlisting}

You can set the optional parameter position specifier to define
 the position of the table, where it should be placed. The following 
characters are all possible placements. Using sequences of it 
define your "wishlist" to LaTeX.

\begin{tabular}[h]{|l|p{15cm}|}
\hline
h &	where the table is declared (here)\\
t &	at the top of the page\\
b & at the bottom of the page\\
p &	on a dedicated page of floats\\
! &	override the default float restrictions. E.g., the maximum size
allowed of a ``b'' float is normally quite small; if you want a large one,
you need this ! parameter as well.\\
\hline
\end{tabular}

Table \ref{tab:myfirsttable} is an example of floating table.
\begin{lstlisting}
\begin{table}[h]
  \centering
  \begin{tabular}{ l | c || r }
    \hline
    1 & 2 & 3 \\ \hline
    4 & 5 & 6 \\ \hline
    7 & 8 & 9 \\
    \hline
  \end{tabular}
  \caption{This table shows some data}
  \label{tab:myfirsttable}
\end{table}
\end{lstlisting}
\begin{table}[h]
  \centering
  \begin{tabular}{ l | c || r }
    \hline
    1 & 2 & 3 \\ \hline
    4 & 5 & 6 \\ \hline
    7 & 8 & 9 \\
    \hline
  \end{tabular}
  \caption{This table shows some data}
  \label{tab:myfirsttable}
\end{table}

\subsection{Floats, Figures and Captions}

\subsubsection{Figures}

To create a figure that floats, use the figure environment.
\begin{lstlisting}
\begin{figure}[placement specifier]
... figure contents ...
\end{figure}
\end{lstlisting}

The placement specifier parameter exists as a compromise, and its
purpose is to give the author a greater degree of control over where 
certain floats are placed. Specifier Permission

\begin{tabular}[h]{|l|p{15cm}|}
\hline
h &	Place the float here, i.e., approximately at the same point it
occurs in the source text (however, not exactly at the spot)\\
t	& Position at the top of the page.\\
b &	Position at the bottom of the page.\\
p &	Put on a special page for floats only.\\
!	& Override internal parameters LaTeX uses for determining "good" float positions.\\
H	& Places the float at precisely the location in the LaTeX code.
Requires the float package, \eg, {\verb \usepackage{float} }. This
is somewhat equivalent to {\verb h! }.\\
\hline
\end{tabular}

\subsubsection{Captions}

It is always good practice to add a caption to any figure or table.
 Fortunately, this is very simple in LaTeX. All you need to do is use 
the {\verb \caption{''text''} } command within the float environment. Because 
of how LaTeX deals sensibly with logical structure, it will
automatically keep track of the numbering of figures, so you do not
need to include this within the caption text.

The location of the caption is traditionally underneath the float. 
However, it is up to you to therefore insert the caption command after 
the actual contents of the float (but still within the
environment). If you place it before, then the caption will appear
above the float. 

Figure \ref{fig:dog} and Figure \ref{fig:reversedog} are two examples.
\begin{lstlisting}
\begin{figure}[h!]
  \caption{A picture of a dog.}
  \centering
    \includegraphics[width=0.5\textwidth,
     natwidth=1024,natheight=768]{dog.jpg}\label{fig:dog}
\end{figure}
\end{lstlisting}

\begin{figure}[h!]
  \caption{A picture of a dog.}
  \centering
    \includegraphics[width=0.5\textwidth,
       natwidth=1024,natheight=768]{dog.jpg}\label{fig:dog}
\end{figure}

\begin{lstlisting}
\begin{figure}[h!]
  \centering
    \reflectbox{ \includegraphics[width=0.5\textwidth,natwidth=1920,natheight=1080]{dog.jpg} }
  \caption{A picture of the same dog
           looking the other way!}
\end{figure}
\end{lstlisting}

\begin{figure}[h!]
  \centering
    \reflectbox{ \includegraphics[width=0.5\textwidth,
        natwidth=1920,natheight=1080]{dog.jpg} }
  \caption{A picture of the same dog
           looking the other way!}\label{fig:reversedog}
\end{figure}

\subsubsection{Multiple Figures in One Float}

A useful extension is the {\verb subcaption } package, which uses
subfloats within a single float. This gives the author the ability to
have subfigures within figures,or subtables within table
floats. Subfloats have their own caption, and an optional global
caption. An example will best illustrate the usage of this package:

\begin{lstlisting}
\begin{figure}[t]
        \centering
        \begin{subfigure}[b]{0.45\textwidth}
                \centering
                \includegraphics[width=\textwidth,natwidth=1024,
                     natheight=768]{dog.jpg}
                \caption{A dog}
                \label{fig:dog2}
        \end{subfigure}%
         \qquad %add desired spacing between images, e. g. ~, \quad, \qquad etc.
          %(or a blank line to force the subfigure onto a new line)
        \begin{subfigure}[b]{0.45\textwidth}
                \centering
                \includegraphics[width=\textwidth,natwidth=1000,
                     natheight=781]{cat.jpg}
                \caption{A cat}
                \label{fig:cat}
        \end{subfigure}
        \caption{Pictures of animals}\label{fig:animals}
\end{figure}
\end{lstlisting}

\begin{figure}[t]
        \centering
        \begin{subfigure}[b]{0.45\textwidth}
                \centering
                \includegraphics[width=\textwidth,natwidth=1024,
                     natheight=768]{dog.jpg}
                \caption{A dog}
                \label{fig:dog2}
        \end{subfigure}%
         \qquad %add desired spacing between images, e. g. ~, \quad, \qquad etc.
          %(or a blank line to force the subfigure onto a new line)
        \begin{subfigure}[b]{0.45\textwidth}
                \centering
                \includegraphics[width=\textwidth,natwidth=1000,
                     natheight=781]{cat.jpg}
                \caption{A cat}
                \label{fig:cat}
        \end{subfigure}
        \caption{Pictures of animals}\label{fig:animals}
\end{figure}





\end{document}


